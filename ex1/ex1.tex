\documentclass[12pt,a4paper,final]{article}

% Use utf-8 encoding for foreign characters
\usepackage[utf8]{inputenc}
\usepackage[greek, finnish]{babel}
\usepackage[T1]{fontenc}

% Setup for fullpage use
\usepackage{fullpage}

% Uncomment some of the following if you use the features
%
% Running Headers and footers
%\usepackage{fancyhdr}

% Multipart figures
%\usepackage{subfigure}

% More symbols
\usepackage[fleqn]{amsmath}
\usepackage{amssymb}
\usepackage{latexsym}
\usepackage{mathtools}

% Surround parts of graphics with box
\usepackage{boxedminipage}

% Package for including code in the document
\usepackage{listings}

% Own definitions
\usepackage{amsfonts}
\usepackage{amsthm}
\usepackage[pdftex,colorlinks]{hyperref}
\usepackage[all]{xy}
\newcommand{\HRule}{\rule{\linewidth}{0.5mm}}

\theoremstyle{definition}
\newtheorem{dfn}{Definition}
\newtheorem{thm}{Lause}
\newtheorem{lem}[thm]{Lemma}
\newtheorem{ex}{Tehtävä}
% \newtheorem{maar}[thm]{Määritelmä}
% \newtheorem{esim}[thm]{Esimerkki}

% \theoremstyle{remark}
% \newtheorem*{huom}{Huomautus}

\newcommand{\N}{\mathbb{N}}
\newcommand{\Z}{\mathbb{Z}}
\newcommand{\Q}{\mathbb{Q}}
\newcommand{\R}{\mathbb{R}}
\newcommand{\C}{\mathbb{C}}
% \newcommand{\P}{\mathbb{P}}

\swapnumbers

% This is now the recommended way for checking for PDFLaTeX:
\usepackage{ifpdf}

%\newif\ifpdf
%\ifx\pdfoutput\undefined
%\pdffalse % we are not running PDFLaTeX
%\else
%\pdfoutput=1 % we are running PDFLaTeX
%\pdftrue
%\fi

\ifpdf
\usepackage[pdftex]{graphicx}
\else
\usepackage{graphicx}
\fi

\usepackage[normalem]{ulem}
\usepackage{enumerate}

\usepackage{titlesec}
% \titleformat{\section}{\normalfont\Large\bfseries}{\S\thesection}{1em}{}
\titleformat{\section}
  {\normalfont\Large\sffamily\bfseries}{\uline{\S\thesection\hspace*{ 1em}}}{0em}{\uline}
\setcounter{section}{-1}

\title{Harjoitus 1}
\author{APKI S13}

\date{\today}

\begin{document}

\ifpdf
\DeclareGraphicsExtensions{.pdf, .jpg, .tif}
\else
\DeclareGraphicsExtensions{.eps, .jpg}
\fi

\maketitle

\begin{ex}
  Hei,
  Nimeni on Timo Sand, olen Tietojenkäsittelytieteen opiskelija. Minulla on noin sivuaineen verran (30op) matikan opintoja tehtynä ja tarvitsen toiset 30op vielä päälle maisterilinjaani varten.
  Valitsin virtuaalisen kurssin koska työn kanssa luennoilla käyminen on haastavaa ja nauhoitetut luennot ovat osoittautuneet minulle toimivaksi konseptiksi.
\end{ex}

\begin{ex}
  "Laskettava $n$ ensimmäistä positiivistä kokonaista lukua yhteen."
  \begin{enumerate}[(a)]
    \item Summakaava : $ \sum_{i=1}^n i$
    \item Induktiotodistus:
      \begin{align*}
        n &= 1 \\
        1 &= \frac{1^2}{2} + \frac{1}{2} \\
        \\
        n &= k \\
        1 + 2 + .. + k &= \frac{k^2}{2} + \frac{k}{2} \\
        \\
        n &= k+1 \\
        1 + 2 + .. + k + (k+1) &= \frac{(k+1)^2}{2} + \frac{k+1}{2} \\
        \frac{k^2}{2} + \frac{k}{2} + (k+1) &= \frac{k^2 + 2k + 1}{2} + \frac{k}{2} + \frac{1}{2} \\
        \frac{k^2}{2} + \frac{k}{2} + (k+1) &= \frac{k^2}{2} + \frac{2k}{2} + \frac{1}{2} + \frac{k}{2} + \frac{1}{2} \\
         (k+1) &= \frac{2k}{2} + \frac{1}{2} + \frac{1}{2} \\
         k + 1 &= k + 1\\
      \end{align*}
  \end{enumerate}
\end{ex}

\begin{ex}
  \begin{enumerate}[(a)]
    \item Summamerkintä: $ \sum_{i=1}^n 2^i$
    \item Induktiotodistus:
      \begin{align*}
        n &= 1\\
        2^1 &= 2^(1+1) - 2\\
        \\
        n &= k \\
        2^1 + 2^2 + .. + 2^k &= 2^{k+1} - 2\\
        \\
        n &= k+1\\
        2^1 + 2^2 + .. + 2^k + 2^{k+1} &= 2^{k+2} - 2\\
        2^{k+1} - 2 + 2^{k+1} &= 2^{k+1} * 2 - 2\\
        2^{k+1} * 2 - 2  &= 2^{k+1} * 2 - 2\\
      \end{align*}
  \end{enumerate}
\end{ex}

\end{document}
